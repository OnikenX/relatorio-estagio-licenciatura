\section*{Resumo}
% \addcontentsline{toc}{section}{\protect\numberline{}Resumo}

No mundo dos jogos de vídeo, têm vindo a ser experimentadas variadas tecnologias, sendo as mesmas posteriormente evoluídas de forma a melhor satisfazerem o objetivo da sua utilização.

A \textit{Nerd Monkeys}, uma empresa desenvolvedora de jogos de vídeo com interesse em explorar novas tecnologias, foi a empresa onde o estágio apresentado neste trabalho se desenvolveu.

A primeira tecnologia usada no estágio foi a \textit{blockchain}, tal como estava planeado na proposta de estagio. 
Relativamente ao \textit{blockchain} foi maioritariamente estudada a tecnologia subjacente, o seu modo de funcionamento, os seus protocolos e as suas aplicações genéricas e concretamente nos jogos de vídeo.

Para experimentar alguns dos conceitos acumulados na pesquisa sobre \textit{blockchain} foi realizado um projeto designado de \textit{AC7ION} do tipo servidor-cliente. Ambos foram desenvolvidos em \textit{Rust}. O servidor faz a troca de informações com a \acrfull{api} do \textit{Twitter}. Este servidor envia os \textit{tweets} obtidos da \acrshort{api} para uma biblioteca partilhada quando requisitados. A biblioteca pode ser usada por qualquer \textit{game engine} que suporte uma \acrfull{abi} para a linguagem \textit{C}.

O último projeto efetuado foi um detetor de batimentos cardíacos com o sensor de \acrfull{ir} da \gls{switch} desenvolvido em \textit{C++}. Este teve como objetivo dar a conhecer o \textit{hardware} e \textit{software} desta consola de jogos de vídeo, permitindo conhecer o espaço de desenvolvimento de uma consola e o que pode ser nela usado e desenvolvido.

\textit{\textbf{Palavras-chave:} Blockchain, Rust, \acrshort{ir} Sensor, Jogos de vídeo}
