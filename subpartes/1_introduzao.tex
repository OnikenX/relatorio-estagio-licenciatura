\chapterf{Introdução}


\label{chap:introducao}

Na área dos jogos de vídeo várias são as tecnologias exploradas de forma a melhorar as experiências dos utilizadores ao jogar.

O trabalho apresentado no presente relatório representa também uma contribuição nesta temática. Consiste numa sumarização do
trabalho desenvolvido, tanto a nível teórico como prático, no estágio curricular no âmbito da unidade curricular de Projeto ou Estágio na empresa
Nerd Monkeys, de 1 de março de 2021 a 18 de julho de 2021, para efeitos de conclusão da
Licenciatura de Engenharia Informática, do \acrfull{isec}.

Neste capítulo, é apresentado o contexto do estágio, os seus objetivos,
entidades envolvidas na realização do mesmo, bem como a estrutura do relatório.

\section{Instituto Superior de Engenharia de Coimbra}

O \acrshort{isec} é uma das
unidades constituintes do \gls{ipc-education}. A
missão do \gls{ipc-education} corresponde à criação, transmissão e difusão de cultura, ciência e tecnologia. Para isso, é ministrada formação de nível superior para o exercício de atividades profissionais no domínio da Engenharia,
promovendo o desenvolvimento da região em que se insere. Foi fundado em
1974, tendo sido adicionado ao \acrshort{ipc-education} em 1988.

O \acrshort{ipc-education} tem uma variada oferta formativa disponível ao nível de mestrados, licenciaturas e \acrfull{ctesp}. O \acrfull{deis} dedica-se, desde 1989, à formação, investigação, desenvolvimento e prestação de serviços na área da Engenharia Informática. O \acrshort{isec} prima pelo caráter prático dos cursos que ministra, traduzido pelo elevado número de aulas laboratoriais, projetos e estágios o que, aliado à qualidade científica e pedagógica, resulta num elevado índice de empregabilidade.
\cite{isec_wiki}

\section{Nerd Monkeys}

A Nerd Monkeys foi fundada em 2013 por Diogo Vasconcelos e Filipe Duarte Pina \cite{nerdmonkeys_interview_eurogamerpt}. A empresa desenvolve e publica jogos, pelo que está sempre a investir em novas ideias para um jogo, tanto em histórias como tecnologias e paradigmas.

Como exemplo de jogos desenvolvidos na \textit{Nerd Monkeys} podem ser citados os jogos  ``\textit{Inspector Zé e Robot Palhaço em: Crime no Hotel Lisboa}'' (2014), ``\textit{Inspector Zé e Robot Palhaço em: O Assassino do Intercidades}'' (2018), ``\textit{Out of Line}'' (2021) e o ``\textit{Monkey Split}'' (2021) \cite{nerd_games_steam} publicados na Steam, uma plataforma de distribuição e venda de jogos.

A empresa também tem um histórico de adaptar jogos para a \gls{switch}, dos quais são exemplos os jogos ``\textit{Inspector Zé e Robot Palhaço em: Crime no Hotel Lisboa}''\cite{ze_eshop}, 
e um jogo de outro programador publicado pela \textit{Nerd Monkeys} cujo o nome do jogo é ``\textit{Traffix}'' \cite{traffix_eshop}. Porém, uma das áreas que a empresa quer explorar é a aplicação de \textit{blockchain} em jogos \textit{online}. Neste sentido, o estágio decorre no contexto da pesquisa de novas tecnologias para usar em novos jogos, neste caso especifico utilizando \textit{blockchain}.

\section{Proposta de Estágio}

A proposta de estágio (\hyperref[anexo:A]{anexo A}) elaborada pela \textit{Nerd Monkeys} visou a investigação da aplicação de \textit{blockchain} em jogos de vídeo, com a adição da utilização do motor de jogo \textit{Unreal Engine} com \textit{C++}.

O planeamento do desenvolvimento ficou divido nas seguintes tarefas:

\begin{itemize}
    \item T1 - Pesquisa e experiência prática – Durante este período o estagiário deverá realizar
    uma pesquisa extensiva sobre todas as componentes envolvidas no projeto e
    experimentação prática para validar os resultados da pesquisa. (5 semanas)

    \item T2 - Criação de ferramentas - tendo em vista as etapas T3 e T4. (4 semanas)

    \item T3 - Criação de protótipo – Criar um protótipo com base na pesquisa desenvolvida com um objetivo definido pelo orientador.(3 semanas)

     \item T4 - Desenvolvimento do videojogo – Com base no protótipo e ferramentas criadas deverá ser desenvolvido um videojogo em, pelo menos, formato de vertical \textit{slice} (demonstração jogável de todas as mecânicas). (10 semanas)

    \item T5 - Elaboração de relatório de estágio. (2 semanas)
\end{itemize}

O planeamento temporal das várias fases é apresentado na \cref{tab:diagrama-temporal}.

\begin{table}[H]
  \centering
  \footnotesize
  \def\arraystretch{2}
  \setlength\tabcolsep{3pt}
  \begin{tabularx}{\textwidth}
  {l*{24}{c}}
  \multicolumn{25}{c}{\textbf{Semanas}}\\
    &
    \multicolumn{1}{c}{01}&
    \multicolumn{1}{c}{02}&
    \multicolumn{1}{c}{03}&
    \multicolumn{1}{c}{04}&
    \multicolumn{1}{c}{05}&
    \multicolumn{1}{c}{06}&
    \multicolumn{1}{c}{07}&
    \multicolumn{1}{c}{08}&
    \multicolumn{1}{c}{09}&
    \multicolumn{1}{c}{10}&
    \multicolumn{1}{c}{11}&
    \multicolumn{1}{c}{12}&
    \multicolumn{1}{c}{13}&
    \multicolumn{1}{c}{14}&
    \multicolumn{1}{c}{15}&
    \multicolumn{1}{c}{16}&
    \multicolumn{1}{c}{17}&
    \multicolumn{1}{c}{18}&
    \multicolumn{1}{c}{19}&
    \multicolumn{1}{c}{20}&
    \multicolumn{1}{c}{21}&
    \multicolumn{1}{c}{22}&
    \multicolumn{1}{c}{23}&
    \multicolumn{1}{c}{24}
    \tabularnewline 
    T1 &
    \cellcolor{blue}{}&\cellcolor{blue}{}&
    \cellcolor{blue}{}&\cellcolor{blue}{}&
    \cellcolor{blue}{}&\\
    T2 &
    &&&&&
    \cellcolor{blue}{}&\cellcolor{blue}{}&
    \cellcolor{blue}{}&\cellcolor{blue}{}&\\
    T3 &
    &&&&&&&&&
    \cellcolor{blue}{}&\cellcolor{blue}{}&
    \cellcolor{blue}{}&\\
    T4 &
    &&&&&&&&&&&&
    \cellcolor{blue}{}&\cellcolor{blue}{}&
    \cellcolor{blue}{}&\cellcolor{blue}{}&
    \cellcolor{blue}{}&\cellcolor{blue}{}&
    \cellcolor{blue}{}&\cellcolor{blue}{}&
    \cellcolor{blue}{}&\cellcolor{blue}{}&\\
    T5 &
    &&&&&&&&&&&&&&&&&&&&&&
    \cellcolor{blue}{}&\cellcolor{blue}{}
  \end{tabularx}
  \caption{Diagrama de Gantt representativo do planeamento temporal}
  \label{tab:diagrama-temporal}
\end{table}

\section{Objetivos e Plano de Trabalhos}

Os objetivos iniciais deste trabalho, consistiam em aferir as capacidades da tecnologia \textit{blockchain} para além da sua utilização comum atual e aplicação concreta em jogos de vídeo com a subsequente criação de um prototipo funcional. Porém, foi considerado de interesse que o trabalho fosse principalmente de pesquisa, dadas as enormes capacidades e potencialidades desta tecnologia e assim aferir quais as melhores opções para a aplicar no contexto da empresa.

\section{Estrutura do relatório}

Este documento, para além da introdução, é composto pelos seguintes capítulos:

\begin{itemize}
    \item \Cref{chap:introducao} - Este é o primeiro capítulo, começa com a descrição da entidade de acolhimento e da instituição de ensino, seguindo-se a apresentação da proposta de estágio, os objetivos definidos e plano de trabalhos. Termina com esta secção que descreve a estrutura do relatório, bem como, a metodologia e notas extras.
    
    \item \Cref{chap:blockchain} - Neste capítulo é realizada a apresentação da primeira parte do estágio, onde foi realizada pesquisa sobre o tema de \textit{blockchain}. É realizada a descrição das várias tecnologias relacionadas com \textit{blockchain}, bem como análises e conclusões de como estas podem ser aplicadas num jogo.

    \item \Cref{chap:ac7ion} - Este capítulo apresenta a segunda parte do estágio, onde é relatado o desenvolvimento de um programa denominado \textit{AC7ION}. A função deste programa é a de receber \textit{tweets} para serem enviados a um cliente que faz comunicação com jogos.

    \item \Cref{chap:bpm} - Este capítulo apresenta a terceira parte do estágio, onde foi desenvolvido um detector de batimentos cardíacos, para a \textit{Nintendo Switch}, que usa o \acrfull{ir} Sensor desta para os detetar.

    \item \Cref{chap:conclusao} - Este capítulo trata da conclusão. No mesmo são tecidas algumas ideias finais sobre o trabalho desenvolvido e é analisado o sucesso do trabalho realizado.
\end{itemize}

Em complemento e como suporte de informação a estes capítulos, são também apresentadas \hyperref[bibliografia]{as referências bibliográficas} e um
 \hyperref[chap:anexos]{anexo}, estes podem ser vistos no final do documento.

\section{Metodologia e Notas Extras}

Os projetos aqui apresentados foram todos desenvolvidos como projetos internos da \textit{Nerd Monkeys}. Em projetos internos da empresa os membros das equipas têm grande liberdade criativa, o que significa que todos têm a liberdade de discutir e mudar o rumo do que está a ser desenvolvido se assim o líder do projeto concordar e for bem justificado.

Sendo que a metodologia de desenvolvimento era ágil e comparável com a \textit{Extreme Programming}, em que existe constante \textit{feedback}, reuniões todos os dias, em que a equipa está em constante contacto.

Os requisitos descritos nos vários capítulos foram inseridos com o propósito de definir os objetivos do trabalho a realizar. Tendo isso em conta, esses requisitos servem apenas como guias/
ideias base, pelo que podem não ser seguidos à risca durante o processo de desenvolvimento.


