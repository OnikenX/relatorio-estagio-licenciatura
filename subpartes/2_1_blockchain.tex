
\section{Introdução}

Como requerido na \hyperref[anexo:A]{proposta de estágio} este trabalho centrou-se na pesquisa de \textit{blockchain}. O objetivo desta pesquisa foi de averiguar a pertinência e possibilidades da área de \textit{blockchain} e do mundo de tecnologias descentralizadas para a área dos jogos de vídeo. Esta pesquisa permitiria concluir se é possível ou não desenvolver jogos descentralizados baseados em \textit{blockchain} em que estes só precisam do poder computacional dos dispositivos dos jogadores para fazer verificação de dados e transições num jogo \textit{online}.

Os jogos feitos com o projeto \gamechaining{} devem ser o menos centralizados possível, reduzindo a utilização de servidores da empresa, permitindo que seja a comunidade do jogo a gerir os servidores do mesmo. Este tipo de independência não permite a utilização de mecanismos de \gls{pay2win}, sendo que a empresa só lucrará com a compra dos conteúdos dos jogos como música, modelos, texturas e outros.

O capítulo divide-se em várias partes, sendo estas:

\begin{itemize}
  \item \underline{\nameref{sec:blockchain_what}} - Nesta secção é explicado o conceito de \textit{blockchain} e termos relacionados.
  
  \item \underline{\nameref{sec:blockchain_where}} - Nesta secção é listada a maneira como esta tecnologia é usada no mercado.
  \item \underline{\nameref{sec:blockchain_how}} - Nesta secção é indicada a forma como a tecnologia \textit{blockchain} pode ser aplicada em jogos de vídeo.
  \item \underline{\nameref{sec:blockchain_conclusao}} - Nesta secção podem ser vistas conclusões sobre os assuntos expostos neste capítulo.
\end{itemize}