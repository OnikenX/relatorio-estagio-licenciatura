
\chapter{Conclusão}
\label{chap:conclusao}
Dando o estágio por terminado, é necessário fazer um balanço do que foi conseguido, tanto na perspetiva
do estagiário como da empresa, e do que será necessário e prioritário desenvolver no futuro para que o projeto desenvolvido evolua. É este o objetivo do presente capítulo.

\section{Objetivos Atingidos}
 
Com a chegada das criptomoedas e a sua exploração e popularidade houve um crescente interesse por \textit{blockchain} e suas áreas de aplicação. 

A \textit{Nerd Monkeys} é uma das várias empresas que tiveram interesse no assunto e fizeram a proposta de estágio para que o estagiário, João Gonçalves, estudasse a matéria em causa e chegasse as suas conclusões sobre ela.

O processo estava a correr bem mas estava a ser mais longo do que o esperado, impossibilitando o cumprimento do plano dentro de previsto. Assim, seria previsível que o estagiário acabasse o estágio sem qualquer implementação feita pelo que se decidiu mudar os planos e propor ao estagiário a realização de outras tarefas que mostrassem a sua capacidade de implementação e que  aplicassem alguns conhecimentos que podiam interessar ao presente estudo.

O primeiro foi o \textit{AC7ION}, este ajudou o estagiário a ganhar experiência com Rust e como fazer bibliotecas para motores de jogo. Houve um período em que se testava qual a forma correta de fazer \acrshort{abi} com \textit{C++} e \textit{Rust} para testar como alguns motores de jogos reagiam e desenvolvendo-se o \textit{AC7ION}.

De seguida desenvolveu se a biblioteca para o \textit{DMS} com bons resultados.

Concluindo, as investigações e desenvolvimentos feitos nesta área durante o estágio, do ponto de vista dos interesses da empresa nesta temática, obteve resultados positivos o que fez com que o estágio fosse um sucesso.

\section{Desenvolvimentos Futuros}

O \gamechaining{} é um projeto que tanto a \textit{Nerd Monkeys} como o estagiário querem continuar por isso será algo que vai se retomar quando considerado oportuno. Enquanto isso o estagiário vai continuar a trabalhar com a \textit{Nerd Monkeys} em outros projetos.